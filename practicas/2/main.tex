\documentclass{IEEEtran}

\usepackage[margin=1.5cm,letterpaper]{geometry}
\usepackage[
spanish,
es-nodecimaldot,
es-tabla
%english
]{babel}

\usepackage[nottoc]{tocbibind}

\usepackage[utf8]{inputenc}

\usepackage{blindtext}
\usepackage{multicol}

\usepackage[default]{comfortaa}
\usepackage[T1]{fontenc}


\usepackage{siunitx}
\sisetup{output-exponent-marker=\ensuremath{\mathrm{E}}}

\usepackage{minted}
\usepackage{hyperref}

\usepackage{xcolor}
\definecolor{LightGray}{gray}{0.9}
\definecolor{DarkGray}{HTML}{191919}
\definecolor{custom}{HTML}{F8F8F8}

\usepackage{caption}
\usepackage{breqn}

\usemintedstyle{default}

\usepackage[ruled,vlined]{algorithm2e}

\renewcommand{\listingscaption}{Código}
\renewcommand\listoflistingscaption{Índice de \listingscaption\@s}

\newenvironment{code}[2][htp]
{\VerbatimEnvironment
  \begin{listing}[#1]
    \centering
    \begin{minted}[
      frame=lines,framesep=1mm,baselinestretch=0.5,breaklines=true,bgcolor=custom,fontsize=\scriptsize
    ]{#2}}{
    \end{minted}
  \end{listing}}

\newenvironment{codeC}[3][htp]
{\VerbatimEnvironment
  \begin{listing}[#1]
    \caption{#3}
    \centering
    \begin{minted}[
      frame=lines,framesep=1mm,baselinestretch=0.5,breaklines=true,bgcolor=custom,fontsize=\scriptsize
    ]{#2}}{
    \end{minted}
  \end{listing}}


\newenvironment{codeCL}[4][htp]
{\VerbatimEnvironment
  \begin{listing}[#1]
    \caption{#3}\label{#4}
    \centering
    \begin{minted}[
      frame=lines,framesep=1mm,baselinestretch=0.5,breaklines=true,bgcolor=custom,fontsize=\scriptsize
    ]{#2}}{
    \end{minted}
  \end{listing}}


\graphicspath{ {../../template/img_common/} {./img}}


\usepackage{subfiles}
\usepackage[backend=biber,style=ieee]{biblatex}
\addbibresource{./main.bib}

\newcommand{\tituloPractica}{Practica 02\\Instalación del software de Oracle}


\begin{document}
\author{Cristian Romero Andrade}
\title{\tituloPractica{}}

\author{\IEEEauthorblockN{Cristian Romero Andrade}}

\date{ }

\begin{titlepage}
  \centering
    \includegraphics[width=0.25\textwidth]{unam_logo}\vspace{0.5cm}

    {\scshape{\Huge Facultad de Ingeniería\par{}}}\vspace{0.25cm}

    {\scshape{\Large Bases de Datos Distribuidas\par{}}}\vfill{}


    {\huge \textbf{\tituloPractica{}}}\vfill{}


    {\Large Alumno\\Romero Andrade Cristian}\vfill{}

      {\large Grupo: 01\par{}}\vfill{}

    {\large Profesor\\Ing.~Jorge Alberto Rodríguez Campos}\vfill{}
    \vfil{}
    {\large Semestre\\\textbf{2022--1}}
    \vfill{}
    {\large Fecha de Entrega\\10 de septiembre de 2021}
    \vfill{}
    \includegraphics[width=0.1\textwidth]{inge_logo}

\end{titlepage}


\maketitle{}
\tableofcontents{}

\dotfill{}

% --- --- --- --- --- --- --- --- --- --- --- --- --- --- --- --- --- --- --- ---
\begin{abstract}
La presente muestra paso a paso la instalación del gestor de bases de datos
\textbf{Oracle Database 19c} en una máquina virtual con \textbf{Oracle Linux 8}.
\end{abstract}

\dotfill{}

\section{Introducción}\label{sec:introduccion}
\IEEEPARstart{E}{l} sistema operativo \textbf{Oracle Linux 8} es perfecto
para instalar \textbf{Oracle Database 19c}, la cual  es la versión actual a largo plazo,
que además proporciona el nivel más alto de estabilidad de la versión y el plazo más
largo para asistencia y corrección de errores.

En esta práctica configuraremos la red de nuestros dispositivos, tanto la máquina anfitriona
como la máquina virtual para que haya una comunicación bidireccional entre ellas. Una vez
configurada se usara el protocolo \texttt{ssh} para usar en su totalidad una interfaz desde
la máquina cliente (la máquina anfitriona o cualquier otra que este en la misma red).

\texttt{ssh} (cliente SSH) es un programa para iniciar sesión en una máquina remota y
para ejecutar comandos en una máquina remota. Su objetivo es proporcionar comunicaciones
cifradas seguras entre dos hosts que no son de confianza a través de una red insegura.
Configuraremos este servicio para poder usar conexiones X11\footnote{X11 es un sistema de ventanas común en entornos UNIX} de la VM al cliente.



\section{Objetivo}\label{sec:objetivo}
Realizar la instalación del software de \textbf{Oracle 19c--19.3} (sin crear aún
la Base de Datos).

% === === === === === === === === === === === === === === === === === === === ===
\section{Contenido}\label{sec:contenido}
\subfile{contenido}

% === === === === === === === === === === === === === === === === === === === ===

\section{Conclusión}\label{sec:conclusion}
Se instaló el software \textbf{Oracle Database 19c} en nuestra VM de manera exitosa,
en este escrito se experimento más con el protocolo \texttt{ssh} en vez de un entorno
``real'', puesto que en un futuro dichas prácticas con estos protocolos servirán para
no usar una máquina física y trabajar con pura red con solo lo necesario, aunque
también se puede implementar un servidor VNC.\@
Se configuro el entorno del usuario y grupos que usa la instalación de ODB19c para no
tener inconvenientes durante el proceso de instalación.
% --- --- --- --- --- --- --- --- --- --- --- --- --- --- --- --- --- --- --- ---
\vfill{}
\listoffigures{}
% \listoftables{}
\listoflistings{}
\nocite{oraclebd2021}
\addcontentsline{toc}{section}{Referencias}
\printbibliography{}
\end{document}
