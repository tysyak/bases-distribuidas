\documentclass{IEEEtran}

\usepackage[margin=1.5cm,letterpaper]{geometry}
\usepackage[
spanish,
es-nodecimaldot,
es-tabla
%english
]{babel}

\usepackage[nottoc]{tocbibind}

\usepackage[utf8]{inputenc}

\usepackage{blindtext}
\usepackage{multicol}

\usepackage[default]{comfortaa}
\usepackage[T1]{fontenc}


\usepackage{siunitx}
\sisetup{output-exponent-marker=\ensuremath{\mathrm{E}}}

\usepackage{minted}
\usepackage{hyperref}

\usepackage{xcolor}
\definecolor{LightGray}{gray}{0.9}
\definecolor{DarkGray}{HTML}{191919}
\definecolor{custom}{HTML}{F8F8F8}

\usepackage{caption}
\usepackage{breqn}

\usemintedstyle{default}

\usepackage[ruled,vlined]{algorithm2e}

\renewcommand{\listingscaption}{Código}
\renewcommand\listoflistingscaption{Índice de \listingscaption\@s}

\newenvironment{code}[2][htp]
{\VerbatimEnvironment
  \begin{listing}[#1]
    \centering
    \begin{minted}[
      frame=lines,framesep=1mm,baselinestretch=0.5,breaklines=true,bgcolor=custom,fontsize=\scriptsize
    ]{#2}}{
    \end{minted}
  \end{listing}}

\newenvironment{codeC}[3][htp]
{\VerbatimEnvironment
  \begin{listing}[#1]
    \caption{#3}
    \centering
    \begin{minted}[
      frame=lines,framesep=1mm,baselinestretch=0.5,breaklines=true,bgcolor=custom,fontsize=\scriptsize
    ]{#2}}{
    \end{minted}
  \end{listing}}


\newenvironment{codeCL}[4][htp]
{\VerbatimEnvironment
  \begin{listing}[#1]
    \caption{#3}\label{#4}
    \centering
    \begin{minted}[
      frame=lines,framesep=1mm,baselinestretch=0.5,breaklines=true,bgcolor=custom,fontsize=\scriptsize
    ]{#2}}{
    \end{minted}
  \end{listing}}


\graphicspath{ {../../template/img_common/} {./img}}


\usepackage{subfiles}
\usepackage[backend=biber,style=ieee]{biblatex}
\addbibresource{./main.bib}

\newcommand{\tituloPractica}{Practica 02\\Instalación del software de Oracle}


\begin{document}
\author{Cristian Romero Andrade}
\title{\tituloPractica{}}

\author{\IEEEauthorblockN{Cristian Romero Andrade}}

\date{ }

\begin{titlepage}
  \centering
    \includegraphics[width=0.25\textwidth]{unam_logo}\vspace{0.5cm}

    {\scshape{\Huge Facultad de Ingeniería\par{}}}\vspace{0.25cm}

    {\scshape{\Large Bases de Datos Distribuidas\par{}}}\vfill{}


    {\huge \textbf{\tituloPractica{}}}\vfill{}


    {\Large Alumno\\Romero Andrade Cristian}\vfill{}

      {\large Grupo: 01\par{}}\vfill{}

    {\large Profesor\\Ing.~Jorge Alberto Rodríguez Campos}\vfill{}
    \vfil{}
    {\large Semestre\\\textbf{2022--1}}
    \vfill{}
    {\large Fecha de Entrega\\10 de septiembre de 2021}
    \vfill{}
    \includegraphics[width=0.1\textwidth]{inge_logo}

\end{titlepage}


\maketitle{}
\tableofcontents{}

\dotfill{}

% --- --- --- --- --- --- --- --- --- --- --- --- --- --- --- --- --- --- --- ---
\begin{abstract}
  En la presente se instala el sistema operativo Oracle Linux, el cual se usará el
  en conjunto con el DBMS de Oracle. Se instalara en un entorno virtual usando
  VirtualBox.
\end{abstract}

\dotfill{}

\section{Introducción}\label{sec:introduccion}
\IEEEPARstart{E}{xisten} varias soluciones de bases de datos donde una base de datos es usualmente
controlada por un sistema de gestión de base de datos (DBMS). En conjunto, los datos
y el DBMS, junto con las aplicaciones que están asociados con ellos, se conocen como
un sistema de base de datos, que a menudo se reducen a solo base de datos.

Para este semestre se usará Oracle, el cual la empresa de mismo nombre es una de los
contribuyentes principales de estas soluciones. A parte de ofrecer un DBMS, la
comunidad de Oracle nos proporciona una distribución optimizada para ejecutar
en plenitud software desarrollado de Oracle: \textbf{Oracle Linux}.

\begin{quote}
  \textit{
    Oracle Linux un entorno operativo abierto y completo, ofrece herramientas de
    virtualización, administración e informática nativa en la nube junto con el sistema
    operativo, en una oferta de soporte unificada. Oracle Linux es compatible en un 100\%
    con los binarios de aplicación de Red Hat Enterprise Linux\cite{Oracle}.
  }
\end{quote}

Si bien se puede instalar en una máquina de manera nativa, por conveniencia del autor, la distribución
se instalará en una máquina virtual usando VitualBox, una solución de Oracle para poder
virtualizar arquitecturas x86 y AMD64/Intel64.

\section{Contenido}\label{sec:contenido}
\subfile{contenido/contenido}

\clearpage{}

\section{Conclusión}\label{sec:conclusion}
Instalamos las distribución Oracle Linux en una máquina virtual, donde aparte se optimizó
para que la distribución se ejecute de mejor forma en el entorno de
\href{https://www.virtualbox.org/}{VirtualBox}. Como extra y por comodidad del alumno se
utilizó un puente de red en la máquina virtual para usar el protocolo \texttt{ssh} como transferencia
de archivos (\texttt{SFTP}) y usar el shell remoto (a parte de montar un directorio del host en la VM).

% --- --- --- --- --- --- --- --- --- --- --- --- --- --- --- --- --- --- --- ---

\newpage{}

\listoffigures{}
% \listoftables{}
% \listoflistings{}

\addcontentsline{toc}{section}{Referencias}
\printbibliography{}
\end{document}
