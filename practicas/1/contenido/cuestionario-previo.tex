\documentclass[../main.tex]{subfiles}
\begin{document}
\begin{enumerate}
  \item Investigar el concepto de \textit{Oracle Unbreakable Enterprise Kernel (UEK)}

        Es un kernel de linux modificado que ofrece ciertas optimizaciones desarrolladas
        en colaboración de Oracle Database, Oracle middleware y equipos de ingeniería de
        hardware de Oracle para garantizar la estabilidad y el rendimiento óptimo para
        las cargas de trabajo empresariales más exigentes\cite{oracle2021}.

        Se recomienda usar este kernel para correr productos Oracle.

  \item ¿Qué relación existe entre RedHat y Oracle Enterprise Linux?

        \textbf{RedHat Enterprise Linux y Oracle Enterprise Linux} son distribuciones
        GNU/Linux. Oracle Linux es una distribución gratuita utilizada principalmente
        por equipos de nivel pequeño a mediano con bases de datos de Oracle
        existentes, mientras que RHEL es el favorito de las empresas de nivel
        empresarial que priorizan la estabilidad y el tiempo de actividad.
  \item Función del comando \texttt{nmtui}

        Es una interfaz de texto basado en curses\footnote{Curses es una biblioteca para el control de terminales sobre sistemas tipo Unix, posibilitando la construcción de una Interfaz para el usuario, para aplicaciones ejecutadas en un terminal} para poder controlar el programa
        NetworkManager, la cual es un programa que proporciona a los sistemas la
        detección y configuración automática para conectarse a la red\cite{archwiki:nm}.
  \item Función del comando \texttt{nmcli}

        Es una interfaz de linea de comandos con el mismo propósito que nmtui\cite{archwiki:nm}.
  \item Utilidad del archivo \texttt{/etc/hosts}

        Asocia direcciones IP con nombres de hosts, una línea por dirección IP.\@Antes de la
        llegada del DNS, la tabla de hosts era la única forma de resolver nombres de host
        en la incipiente Internet.
  \item Utilidad el archivo \texttt{/etc/hostname}

        Este archivo sirve para identificar una máquina en una red.
  \item Utilidad del archivo \texttt{/etc/inittab}

        Este archivo indica el modo en el que el sistema será iniciado. Los niveles son indicados
        mediante los números del 0 al 6. En la mayoría de los Unix y distribuciones Linux (aunque
        no podría decir que en todas) el nivel de ejecución son los siguientes:

        \begin{code}
          \begin{minted}{bash}
# Nivel de ejecución predeterminado. Los niveles de ejecución utilizados son:
# 0 - detener (NO establezca initdefault en esto)
# 1 - Modo de usuario único
# 2 - Multiusuario, sin NFS (Lo mismo que 3, si no tiene red)
# 3 - Modo multiusuario completo
# 4 - sin usar
# 5 - X11
# 6 - reiniciar (NO establezca initdefault en esto)

          \end{minted}
        \end{code}

  \item Considerar la siguiente instrucción \texttt{grub2-mkconfig > /dev/null}
        ¿Qué significa la instrucción después del comando \texttt{grub2-mkconfig}?

  \item Investigar las principales características y usos de un \textit{database link}
\end{enumerate}
\end{document}
