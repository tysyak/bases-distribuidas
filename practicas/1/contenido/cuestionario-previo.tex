\documentclass[../main.tex]{subfiles}
\begin{document}
\begin{enumerate}
  \item Investigar el concepto de \textit{Oracle Unbreakable Enterprise Kernel (UEK)}\\

        Es un kernel de linux modificado que ofrece ciertas optimizaciones desarrolladas
        en colaboración de Oracle Database, Oracle middleware y equipos de ingeniería de
        hardware de Oracle para garantizar la estabilidad y el rendimiento óptimo para
        las cargas de trabajo empresariales más exigentes\cite{oracle2021}.

        Se recomienda usar este kernel para correr productos Oracle.\\

  \item ¿Qué relación existe entre RedHat y Oracle Enterprise Linux?\\

        \textbf{RedHat Enterprise Linux y Oracle Linux} son distribuciones
        GNU/Linux. Oracle Linux es una distribución gratuita utilizada principalmente
        por equipos de nivel pequeño a mediano con bases de datos de Oracle
        existentes, mientras que RHEL es el favorito de las empresas de nivel
        empresarial que priorizan la estabilidad y el tiempo de actividad.

        Oracle es una distribución basada de RedHat y este último de Fedora\cite{dw}.\\

  \item Función del comando \texttt{nmtui}\\

        Es una interfaz de texto basado en curses\footnote{Curses es una biblioteca para el control de terminales sobre sistemas tipo Unix, posibilitando la construcción de una Interfaz para el usuario, para aplicaciones ejecutadas en un terminal} para poder controlar el programa
        NetworkManager, la cual es un programa que proporciona a los sistemas la
        detección y configuración automática para conectarse a la red\cite{archwiki:nm}.
  \item Función del comando \texttt{nmcli}\\

        Es una interfaz de linea de comandos con el mismo propósito que \texttt{nmtui}\cite{archwiki:nm}.\\

  \item Utilidad del archivo \texttt{/etc/hosts}\\

        Asocia direcciones IP con nombres de hosts, una línea por dirección IP.\@ Antes de la
        llegada del DNS, la tabla de hosts era la única forma de resolver nombres de host
        en la incipiente Internet.\\

  \item Utilidad el archivo \texttt{/etc/hostname}\\

        Este archivo sirve para identificar una máquina en una red\cite{archwiki:domain}.\\

  \item Utilidad del archivo \texttt{/etc/inittab}\\

        Este archivo indica el modo en el que el sistema será iniciado. Los niveles son indicados
        mediante los números del 0 al 6. En la mayoría de los Unix y distribuciones Linux (aunque
        no podría decir que en todas) el nivel de ejecución son los siguientes:

        \begin{code}{bash}
# Nivel de ejecución predeterminado. Los niveles de ejecución utilizados son:
# 0 - detener (NO establezca initdefault en esto)
# 1 - Modo de usuario único
# 2 - Multiusuario, sin NFS (Lo mismo que 3, si no tiene red)
# 3 - Modo multiusuario completo
# 4 - sin usar
# 5 - X11
# 6 - reiniciar (NO establezca initdefault en esto)
        \end{code}

  \item Considerar la siguiente instrucción \texttt{grub2-mkconfig > /dev/null}
        ¿Qué significa la instrucción después del comando \texttt{grub2-mkconfig}?\\

        Genera el archivo de configuración GRUB\footnote{GRand Unified Bootloader,  es
        un cargador de arranque . El GRUB actual también se conoce como GRUB 2. El GRUB original, o
        GRUB Legacy , corresponde a las versiones 0.9x.} pero el flujo de salida se suprime\cite{archwiki:grub}.\\

  \item Investigar las principales características y usos de un \textit{database link}.\\

        Un enlace de base de datos le permite hacer referencia a objetos en una base de datos remota.
        Normalmente, la base de datos remota será otra base de datos de Oracle, pero puede ser cualquier
        base de datos compatible con ODBC.\@
        Al acceder a un objeto remoto a través de un enlace de base de datos, la base de datos local
        actúa como un cliente de Oracle. Hay una variedad de variaciones de sintaxis en la documentación,
        pero las que usará con más frecuencia son las siguientes.
        \begin{code}{sql}
-- Usuario Remoto: scott
-- Contraseña Remota: tiger

-- Enlace de base de datos privada a un usuario en una base de datos remota.
CREATE DATABASE LINK scott_remote
   CONNECT TO scott IDENTIFIED BY tiger
   USING 'remote_database';

-- Enlace de Base de DAtos a un usuario en una base de datos remota, con cadena completa de conexión.
CREATE DATABASE LINK scott_remote
   CONNECT TO scott IDENTIFIED BY tiger
   USING '(DESCRIPTION=
   (ADDRESS=(PROTOCOL=TCP)(HOST=server1.example.com)
   (PORT=1521))
   (CONNECT_DATA=(SERVICE_NAME=HRDEV1))
   )';

-- Enlace de Base de Datos pública.
CREATE PUBLIC DATABASE LINK scott_remote
   CONNECT TO scott IDENTIFIED BY tiger
   USING 'remote_database';

-- Enlace de Base de Datos privada a un usuario de base de datos local.
CREATE DATABASE LINK scott_local
   CONNECT TO scott IDENTIFIED BY tiger
   USING 'local';
        \end{code}

        Los enlaces de bases de datos privadas solo son visibles para el propietario
        del enlace. Los enlaces de bases de datos públicas son visibles para todos los
        usuarios de la base de datos y, como tales, son una posible pesadilla de
        seguridad.

        Normalmente un enlace de base de datos se usa para conectarse
        a un usuario en una base de datos remota, donde la cláusula \texttt{USING} apunta a una
        entrada en el archivo \texttt{tnsnames.ora} de los servidores de la base de datos. Para
        los enlaces locales, se utiliza la entrada especial de \texttt{local}.

\end{enumerate}
\end{document}
